
\title{Hello World!}
%\author{}
\date{}

\begin{document}

From the top-level directory, run:
\begin{code}
make
./bin/tex2web < example/hello.tex > hello.html
\end{code}
Then open up \ilfile{hello.html} in a browser.
You should see this file as clean and minimal HTML.

Alternatively, one can check the author's university web page for an example of this tool's output:
\url{http://www.csl.mtu.edu/~apklinkh/}

\section{Dependencies}

The main dependencies of \ilname{tex2web} are the author's C utility library found
\href{http://github.com/grencez/cx}{here}
and the preprocessed version
\href{http://github.com/grencez/cx-pp}{here}.
But don't worry about that, the \ilcode{make} command does all the downloading for you.

\section{Features}

This tool does simple LaTeX formatting.
Formats like \textbf{bold}, \textit{italic}, and \texttt{teletype} are supported. as well as some custom macros.

\begin{itemize}
\item \ilcode{inline code}
\item Purpose-built formatting for \ilflag{-command-line-flags}, \ilfile{filenames.h}, \ilsym{symbols}, \illit{LITERAL_VALUES}, \ilname{tool-names}, and \ilkey{keywords}.
 \begin{itemize}
 \item Some of these may look the same, but it's nice to be explicit.
 \end{itemize}
\end{itemize}

\quicksec{Quick Aside}
Sometimes you want to make a quick section without a formal number or gigantic spacing.
In this case, use \ilcode{\quicksec}.

\end{document}

